\documentclass{beamer}
\usepackage[utf8]{inputenc}
\usepackage[spanish]{babel}
\usepackage{xcolor}
\usepackage{listings}
\usepackage{graphicx}
\usepackage{hyperref}

\usetheme{Madrid}
\usecolortheme{default}

% Define colors
\definecolor{astroBlue}{RGB}{85, 100, 255}
\definecolor{darkGray}{RGB}{32, 33, 35}
\definecolor{lightGray}{RGB}{242, 242, 242}

\setbeamercolor{primary}{bg=astroBlue, fg=white}
\setbeamercolor{background canvas}{bg=white}
\setbeamertemplate{navigation symbols}{}

% Code styling
\lstset{
  basicstyle=\ttfamily\small,
  breaklines=true,
  keywordstyle=\color{astroBlue},
  commentstyle=\color{gray},
  stringstyle=\color{red},
  showstringspaces=false,
  backgroundcolor=\color{lightGray},
  frame=single,
  rulecolor=\color{darkGray}
}

\title{Astro: Arquitectura de Islas para Aplicaciones Web Modernas}
\author{Clase Magistral de Desarrollo Web}
\date{\today}

\begin{document}

% Slide 1: Title
\frame{\titlepage}

% Slide 2: Agenda
\begin{frame}
  \frametitle{Agenda}
  \begin{itemize}
    \item ¿Qué es Astro y por qué importa?
    \item Arquitectura de Islas: Concepto Fundamental
    \item Componentes Estáticos vs. Interactivos
    \item Client Directives: Hidratación Selectiva
    \item Caso de Estudio: Dashboard de Gestión de Proyectos
    \item Performance y Mejores Prácticas
    \item Conclusiones y Próximos Pasos
  \end{itemize}
\end{frame}

% Slide 3: What is Astro
\begin{frame}
  \frametitle{¿Qué es Astro?}
  
  Astro es un framework web moderno que se enfoca en el rendimiento mediante un enfoque revolucionario:
  
  \begin{itemize}
    \item \textbf{HTML-First}: Renderiza a HTML estático por defecto
    \item \textbf{Zero JavaScript}: Sin JavaScript innecesario
    \item \textbf{Partial Hydration}: Hidratación selectiva de componentes
    \item \textbf{Framework Agnostic}: Soporta React, Vue, Svelte, etc.
  \end{itemize}
  
  \vspace{0.5cm}
  
  \textit{``Astro ayuda a construir sitios web más rápidos con menos JavaScript.''} -- Documentación oficial de Astro
\end{frame}

% Slide 4: The Problem with SPAs
\begin{frame}
  \frametitle{El Problema con las SPAs Tradicionales}
  
  Las Single Page Applications (SPAs) envían mucho JavaScript al cliente:
  
  \begin{center}
    \begin{tabular}{|c|c|c|}
      \hline
      \textbf{Métrica} & \textbf{SPA Tradicional} & \textbf{Astro} \\
      \hline
      JavaScript Inicial & 200-500 KB & 0-50 KB \\
      \hline
      Time to Interactive & 3-5 segundos & 0.5-1 segundo \\
      \hline
      Rendimiento Móvil & Pobre & Excelente \\
      \hline
    \end{tabular}
  \end{center}
  
  \vspace{0.5cm}
  
  Astro resuelve esto mediante la Arquitectura de Islas.
\end{frame}

% Slide 5: Islands Architecture Concept
\begin{frame}
  \frametitle{Arquitectura de Islas: El Concepto}
  
  La Arquitectura de Islas es un patrón de diseño donde:
  
  \begin{itemize}
    \item La mayoría de la página es HTML estático (agua)
    \item Componentes interactivos son ``islas'' independientes
    \item Cada isla se hidrata de forma selectiva
    \item Las islas no necesitan comunicarse entre sí
  \end{itemize}
  
  \vspace{0.5cm}
  
  \textbf{Ventajas:}
  \begin{itemize}
    \item Rendimiento: Menos JavaScript
    \item Escalabilidad: Islas independientes
    \item Flexibilidad: Múltiples frameworks en la misma página
  \end{itemize}
\end{frame}

% Slide 6: Static Components
\begin{frame}[fragile]
  \frametitle{Componentes Estáticos: HTML Puro}
  
  Por defecto, todos los componentes de Astro se renderan como HTML:
  
  \begin{lstlisting}[language=jsx]
// Este componente se renderiza como HTML puro
export default function ProjectCard({ project }) {
  return (
    <div className="card">
      <h3>{project.name}</h3>
      <p>{project.description}</p>
      <a href={`/projects/${project.id}`}>
        Ver detalles
      </a>
    </div>
  );
}
  \end{lstlisting}
  
  \textbf{Resultado}: HTML rápido, sin JavaScript.
\end{frame}

% Slide 7: Client Directives
\begin{frame}[fragile]
  \frametitle{Client Directives: Agregando Interactividad}
  
  Para hacer un componente interactivo, usas directivas \texttt{client:*}:
  
  \begin{lstlisting}[language=jsx]
// client:load - Carga inmediatamente
<ProjectFilter client:load onChange={handleChange} />

// client:idle - Carga cuando el navegador está inactivo
<ProjectFilter client:idle onChange={handleChange} />

// client:visible - Carga solo cuando es visible
<ProjectFilter client:visible onChange={handleChange} />
  \end{lstlisting}
  
  Cada directiva controla cuándo y cómo se hidrata el componente.
\end{frame}

% Slide 8: Hydration Strategies
\begin{frame}
  \frametitle{Estrategias de Hidratación}
  
  \begin{center}
    \begin{tabular}{|l|p{6cm}|}
      \hline
      \textbf{Directiva} & \textbf{Cuándo Usar} \\
      \hline
      \texttt{client:load} & Componentes críticos que necesitan interactividad inmediata \\
      \hline
      \texttt{client:idle} & Componentes secundarios, se cargan cuando el navegador está inactivo \\
      \hline
      \texttt{client:visible} & Componentes en el viewport, se cargan cuando son visibles \\
      \hline
      \texttt{client:only} & Solo renderiza en cliente (sin HTML estático) \\
      \hline
    \end{tabular}
  \end{center}
  
  \vspace{0.5cm}
  
  \textbf{Principio}: Usa la directiva menos agresiva que satisfaga tus necesidades.
\end{frame}

% Slide 9: Case Study - Dashboard Overview
\begin{frame}
  \frametitle{Caso de Estudio: Dashboard de Gestión de Proyectos}
  
  Hemos construido una aplicación real que demuestra los conceptos de Astro:
  
  \begin{itemize}
    \item \textbf{Componentes Estáticos}: ProjectCard, DashboardStats
    \item \textbf{Client Islands}: DashboardLayout, ProjectFilter
    \item \textbf{Páginas}: Home, ProjectDetail, Tasks
    \item \textbf{Datos}: Mock data para demostración
  \end{itemize}
  
  \vspace{0.5cm}
  
  Esta aplicación es un caso de uso realista que muestra cómo Astro puede mejorar el rendimiento sin sacrificar la funcionalidad.
\end{frame}

% Slide 10: Architecture Diagram
\begin{frame}
  \frametitle{Arquitectura del Dashboard}
  
  \begin{center}
    \textbf{Estructura de Componentes}
    
    \vspace{0.5cm}
    
    \begin{tabular}{|c|c|}
      \hline
      \textbf{Componente} & \textbf{Tipo} \\
      \hline
      DashboardLayout & Client Island (client:load) \\
      \hline
      ProjectCard & Estático \\
      \hline
      DashboardStats & Estático \\
      \hline
      ProjectFilter & Client Island (client:idle) \\
      \hline
    \end{tabular}
  \end{center}
\end{frame}

% Slide 11: Performance Benefits
\begin{frame}
  \frametitle{Beneficios de Rendimiento}
  
  Astro logra mejor rendimiento mediante:
  
  \begin{itemize}
    \item \textbf{Menos JavaScript}: Solo lo necesario se envía al cliente
    \item \textbf{Carga Paralela}: Las islas se cargan de forma independiente
    \item \textbf{Renderizado Rápido}: HTML estático se sirve inmediatamente
    \item \textbf{Mejor SEO}: HTML completo disponible para crawlers
  \end{itemize}
  
  \vspace{0.5cm}
  
  \textbf{Resultado}: Sitios web más rápidos, mejor experiencia de usuario, mejor SEO.
\end{frame}

% Slide 12: Best Practices
\begin{frame}
  \frametitle{Mejores Prácticas con Astro}
  
  \begin{enumerate}
    \item \textbf{Comienza con Estático}: Renderiza como HTML por defecto
    \item \textbf{Añade Interactividad Selectivamente}: Solo donde sea necesario
    \item \textbf{Elige la Directiva Correcta}: client:idle para no-crítico, client:load para crítico
    \item \textbf{Mantén Islas Pequeñas}: Componentes independientes y reutilizables
    \item \textbf{Monitorea el Rendimiento}: Mide el impacto de cada isla
  \end{enumerate}
\end{frame}

% Slide 13: Comparison with Other Frameworks
\begin{frame}
  \frametitle{Astro vs. Otros Frameworks}
  
  \begin{center}
    \begin{tabular}{|l|c|c|c|}
      \hline
      \textbf{Característica} & \textbf{Astro} & \textbf{Next.js} & \textbf{Vue} \\
      \hline
      HTML Estático por Defecto & \checkmark & Parcial & $\times$ \\
      \hline
      Hidratación Selectiva & \checkmark & $\times$ & $\times$ \\
      \hline
      Framework Agnostic & \checkmark & $\times$ & $\times$ \\
      \hline
      Curva de Aprendizaje & Baja & Media & Media \\
      \hline
    \end{tabular}
  \end{center}
\end{frame}

% Slide 14: Real World Applications
\begin{frame}
  \frametitle{Aplicaciones del Mundo Real}
  
  Astro es ideal para:
  
  \begin{itemize}
    \item \textbf{Sitios de Contenido}: Blogs, documentación, landing pages
    \item \textbf{Dashboards}: Como el que hemos construido
    \item \textbf{E-commerce}: Catálogos de productos con carrito interactivo
    \item \textbf{Aplicaciones Híbridas}: Mezcla de contenido estático e interactivo
  \end{itemize}
  
  \vspace{0.5cm}
  
  \textbf{Nota}: Astro no es ideal para aplicaciones completamente interactivas (como editores de código o aplicaciones en tiempo real).
\end{frame}

% Slide 15: Getting Started with Astro
\begin{frame}[fragile]
  \frametitle{Comenzando con Astro}
  
  \begin{lstlisting}[language=bash]
# Crear un nuevo proyecto
npm create astro@latest

# Instalar dependencias
npm install

# Iniciar servidor de desarrollo
npm run dev

# Compilar para producción
npm run build
  \end{lstlisting}
\end{frame}

% Slide 16: Conclusion
\begin{frame}
  \frametitle{Conclusión}
  
  \begin{itemize}
    \item Astro revoluciona cómo construimos sitios web modernos
    \item La Arquitectura de Islas es el futuro del desarrollo web
    \item Rendimiento y funcionalidad no son mutuamente excluyentes
    \item Astro permite construir aplicaciones rápidas y escalables
  \end{itemize}
  
  \vspace{0.5cm}
  
  \textbf{Próximos Pasos}:
  \begin{itemize}
    \item Experimenta con Astro en tus proyectos
    \item Explora la documentación oficial
    \item Únete a la comunidad de Astro
  \end{itemize}
\end{frame}

% Slide 17: Q&A
\begin{frame}
  \frametitle{Preguntas y Respuestas}
  
  \begin{center}
    \Large
    ¿Preguntas?
    
    \vspace{1cm}
    
    Documentación: \url{https://docs.astro.build}
    
    GitHub: \url{https://github.com/withastro/astro}
  \end{center}
\end{frame}

\end{document}
